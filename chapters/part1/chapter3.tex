\chapter{Dynamic Array}

C styled arrays are static, meaning that they have a fixed size. In this chapter, we will learn how to implement a dynamic array, which is a data structure that can grow and shrink in size.

\begin{verbatim}
class DynamicArray {
    capacity: integer  # room for elements
    size:     integer  # actual number of elements
};
\end{verbatim}

\begin{minipage}[t]{0.4\linewidth} \begin{algorithm}[H] \begin{algorithmic}[1]
    \label{algo:da_insert}
    \Procedure{Insert}{$A, x$}
        \If{$A.size = A.capacity$}
            \State $A \gets \textsc{Resize}(A)$
        \EndIf
        \State $A.size \gets A.size + 1$
        \State $A[A.size] \gets x$
    \EndProcedure
\end{algorithmic} \end{algorithm} \end{minipage}
\hfill
\begin{minipage}[t]{0.5\linewidth} \begin{algorithm}[H] \begin{algorithmic}[1]
    \Procedure{Resize}{$A$}
        \State $B \gets \textsc{DynamicArray}(2 \times A.capacity)$
        \For {$i = 1$ \textbf{to} $A.size$}
            \State \textsc{Insert}($B, A[i]$)
        \EndFor
        \State \Return $B$
    \EndProcedure
\end{algorithmic} \end{algorithm} \end{minipage}

{~~~}

To analyze the running time of the above algorithm, see \hyperref[exam:amortized_accounting]{this example} using accounting method for amortized analysis. The amortized running time of the above algorithm is $\Theta(1)$.