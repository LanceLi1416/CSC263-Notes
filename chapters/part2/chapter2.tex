\chapter{Amortized Analysis}

\section{Aggregated Method}

In the \term{aggregated method}\index{Amortized Analysis!Aggregated Method}, we determine the upper bound $T(n)$ on the total cost of a sequence of $N$ operations, then calculate the average cost per operation as $\frac{T(n)}{n}$. %This is the same as the \term{average-case analysis} that we have been doing so far.

\section{Accounting Method}

The \term{accounting method}\index{Amortized Analysis!Accounting Method} is a form of aggregate analysis which assigns to each operation an amortized cost which may differ from its actual cost. Early operations have an amortized cost higher than their actual cost, which accumulates a saved ``credit'' that pays for later operations having an amortized cost lower than their actual cost. Because the credit begins at zero, the actual cost of a sequence of operations equals the amortized cost minus the accumulated credit. Because the credit is required to be non-negative, the amortized cost is an upper bound on the actual cost. Usually, many short-running operations accumulate such credit in small increments, while rare long-running operations decrease it drastically.